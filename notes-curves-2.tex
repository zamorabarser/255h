\documentclass{ximera}

\title{Line integrals of vector fields}
\license{CC: 0}         % replace with an appropriate license, or set it in xmPreamble

\begin{document}

\begin{abstract}
    Line integrals of vector fields
\end{abstract}

\section{Oriented curves}

An oriented curve is a curve with a choice of direction. Each curve has two orientations: forward and backward. A parametrization of an oriented curve is a parametrization that travels the curve in the correct direction. 


Note: the words ``forward'' and ``backward'' are subjective. 

When a plane curve $\gamma : [a,b] \to \mathbb{R}^2 $ is closed and simple ($\gamma (a) = \gamma (b)$, and the curve doesn't self intersect), then we say it is positively oriented if it is oriented with counterclockwise direction and negatively oriented if it is oriented with clockwise direction. 

For example, the curve 
\[    \gamma (t) = (\cos t , \sin t)               \]
with $0 \leq t \leq 2 \pi$ is positively oriented, while the curve
\[     \sigma (t) = (  \sin t, \cos t  )                   \]
with $0 \leq t \leq 2 \pi$ is negatively oriented.
\section{Line integrals of vector fields}


\begin{definition}
    Let $C$ be an oriented curve,  $\gamma: [a,b] \to \mathbb{R}^3$ a parametrization, and 
    \[    F (x,y,z) = \langle  P(x,y,z) , Q(x,y,z) , R(x,y,z)   \rangle         \]
    a vector field. The integral of $F$ along $C$ is given by
    \[  \int_C F \cdot ds  : = \int_C (P \, dx + Q\,  dy + R \, dz) : = \int_a^b F(\gamma (t) ) \cdot \gamma '(t) \, dt  .                \]
\end{definition}

Note: if you use a parametrization in the opposite direction, the integral changes sign. 

If we let 
\[ T(t) : = \frac{\gamma ' (t) }{ \vert \gamma ' (t) \vert }    \]
be the unit tangent vector, then the line integral above becomes
\begin{align*}
    \int_C F \cdot ds & = \int_a ^b F(\gamma (t)) \cdot \gamma ' (t) \, dt \\
    & = \int_a ^b F(\gamma (t)) \cdot \frac{ \gamma ' (t)}{\vert \gamma ' (t) \vert } \, \vert \gamma ' (t) \vert  \, dt \\
    & = \int_C \left[   F  \cdot T          \right] \, ds 
\end{align*}

The dot product $F \cdot T$ represents ``how much is $F$ pointing in the direction in which $\gamma$ is moving''.  Therefore the integral can be interpreted as ``how much did the force field $F$ help $\gamma$ perform its trajectory''.

\begin{example}
    Let $F  = \langle 1,0 \rangle$ and $\alpha : [0,1] \to \mathbb{R}^2$, $\beta : [0,1] \to \mathbb{R}^2$, $\gamma : [0,1] \to \mathbb{R}^2$ be given by 
    \begin{align*}
        \alpha (t) & = (t,0) \\
        \beta (t) & = (0,t) \\
        \gamma (t) & = (-t,0) 
    \end{align*}
    This means: 
    \begin{center}
        $\alpha$ is moving right

        $\beta$ is moving up

        $\gamma $ is moving left
    \end{center}
    Since $F$ is pointing right, it is helping $\alpha $ perform its trajectory, it is not helping nor preventing $\beta$ from performing its trajectory, and is pushing $\gamma$ against its trajectory. From here we intuitively deduce that 
    \begin{center}
        $\int_{\alpha} F \, ds$ is positive

        
        $\int_{\beta} F \, ds$ is zero

        
        $\int_{\gamma} F \, ds$ is negative
    \end{center}
    This can be easily computed from the dot products:
    \begin{align*}
        F(\alpha (t)) \cdot \alpha ' (t) & = \langle 1,0 \rangle \cdot \langle 1,0 \rangle = 1\\
        F(\beta (t)) \cdot \beta ' (t) & = \langle 1,0 \rangle \cdot \langle 0,1 \rangle = 0\\
        F(\gamma (t)) \cdot \gamma ' (t) & = \langle 1,0 \rangle \cdot \langle -1,0 \rangle = -1
    \end{align*}
    Basically:
    \begin{itemize}
        \item if $\alpha$ goes with the flow of $F$, the integral $\int_{\alpha} F\cdot ds$ is positive.
        \item if $\gamma $ is swimming against the current, the integral $\int_{\gamma} F \cdot ds$ is negative.
    \end{itemize}

\end{example}

The integral $\int_C F \cdot ds$ is also called the work of $F$ along the trajectory. 

\section{Exercises on line integrals}

\begin{exercise}
    Let $F (x,y,z) = \langle  - 2y , z^2 + 3x  , x - 1  \rangle$ and $C$ the curve with parametrization $\gamma : [0,2 ] \to \mathbb{R}^3$ given by
    \[      \gamma (t) = (  t^2 - 3, 2 t , t^3  ).                      \]
    Find 
    \[    \int_C F \cdot ds      \]
\end{exercise}

Using the change of variables 
\begin{align*}
    x(t) & = t^2 - 3\\
    y(t) & = 2t \\
    z(t) & = t^3
\end{align*}
we get
\[     F (\gamma (t) )   = \langle -4t  , t^6 + 3t^2 - 9 , t^2 - 4 \rangle             \]
We also need
\[ \gamma ' (t) = \langle  2t , 2, 3t^2 \rangle         \] 
Then
\begin{align*}
\int_C F \cdot ds   & = \int_0 ^2  \langle -4t  , t^6 + 3t^2 - 9 , t^2 - 4 \rangle   \cdot \langle 2t , 2, 3t^2 \rangle \, dt \\
& = \int _0 ^2 \left[   - 8 t^2 + 2t^6 + 6 t^2 - 18 + 3 t^4 - 12 t^2      \right] \, dt \\
&= - \frac{64}{3}   + \frac{256}{7} + \frac{48}{3} - 36 + \frac{96}{5} - 32       
\end{align*}




\begin{exercise}
    Let $F (x,y,z) = \langle  z + 1 ,  x , y   \rangle$ and $C$ the curve with parametrization $\gamma : [0, 3] \to \mathbb{R}^3$ given by
    \[      \gamma (t) = (  e^{t}  , -  t^2  ,   t     ).               \]
    Find 
    \[    \int_C F \cdot ds      \]
\end{exercise}

Using the change of variables 
\begin{align*}
    x(t) & = e^t   \\
    y(t) & =  - t^2    \\
    z(t) & = t
\end{align*}
we get
\[     F (\gamma (t) )   = \langle   t + 1 , e^t , - t^2   \rangle             \]
We also need
\[ \gamma ' (t) = \langle  e^t , - 2t , 1  \rangle         \] 
Then
\begin{align*}
\int_C F \cdot ds   & = \int_0 ^3  \langle  t + 1 , e^t , - t^2   \rangle   \cdot \langle  e^t , - 2t , 1  \rangle \, dt \\
& = \int _ 0 ^3  \left[  te^t + e^t  - 2 t e^t - t^2       \right] \, dt \\
& = \left[   -t e^t + 2e^t - t^3 / 3       \right] \vert _{t =0} ^3 \\
& = -3 e^3 + 2 e ^3 - 9 - 2 \\
& = - e ^3 - 11 
\end{align*}


\section{Fundamental Theorem of Calculus II}

The classic Fundamental Theorem of Calculus says that the integral of the derivative of a function $F(x)$ is the function $F$ itself:
\[   \int_a^b F' (x) \, dx  = F(b) - F(a)             \]

Something similar happens when we take the line integral of a gradient. Consider a differentiable function $f  : D \to \mathbb{R}$ with $D \subset \mathbb{R}^3$ and its gradient vector field  $  \nabla f $. Also take an oriented curve $C \subset \mathbb{R}^3$ and a parametrization $\gamma : [a,b] \to \mathbb{R}^3$. Recall that by the chain rule, we have
\[      \frac{d}{dt} (f ( \gamma (t)) ) = \nabla f (\gamma (t) ) \cdot \gamma ' (t)    .     \]
Therefore
\begin{align*}
    \int_C \nabla f \cdot ds & = \int_a^b \nabla f (\gamma (t)) \cdot \gamma ' (t) \, dt \\
    & = \int _a^b  \frac{d}{dt} ( f ( \gamma (t) ) ) \, dt \\
    & =  f (\gamma (b)) - f (\gamma (a)) 
\end{align*}

\begin{theorem}
    Let $D \subset \mathbb{R}^3$ be a region, $f : D \to \mathbb{R}^3$ a differentiable function,  $C \subset \mathbb{R}^3$ an oriented curve with parametrization $\gamma : [a,b] \to \mathbb{R}^3$. Then
    \[         \int_C \nabla f \cdot ds  = f (  \gamma (b)  )  - f(\gamma (a))                           \]
    In particular, the integral $\int_C \nabla f \cdot ds$ does only depend on the endpoints of $C$ and not on the trajectory. 
\end{theorem}




\begin{exercise}
    Let $F (x,y,z) = \langle  x , \cos y , e^z   \rangle$ and $C$ the curve with parametrization $\gamma : [0, \pi ] \to \mathbb{R}^3$ given by
    \[      \gamma (t) = (   t^2 \sqrt{ t^2 + 1 } , e^{t^2} ,  t^2 \cos t    ).               \]
    Find 
    \[    \int_C F \cdot ds      \]
\end{exercise}

Note that 
\[      \text{curl}(F) = \langle 0,0,0 \rangle   ,                         \]
and the domain of $F$ is $\mathbb{R}^3$, so $F$ is conservative. To find the potential function, we do partial integration, 
\begin{align*}
    f(x,y,z) & = x^2 / 2 + g_1 (y,z) \\
    f(x,y,z) & = \sin y + g_2 (x,z) \\
    f(x,y,z) & = e^z + g_3(x,y)
\end{align*}
Matching terms, we get
\[       f (x,y,z) = x^2 /2  + \sin y + e^z + C                        \]
On the other hand, the endpoints of $C$ are
\begin{align*}
    \gamma (0) & = (   0, 1, 0 ) \\
    \gamma (\pi ) & = (  \pi ^2 \sqrt{\pi ^2 + 1 } , e ^{\pi ^2} , - \pi ^2  )
\end{align*}
Therefore, 
\begin{align*}
    \int_C F \cdot ds & = f(   \pi ^2 \sqrt{\pi ^2 + 1 } , e ^{\pi ^2} , - \pi ^2    ) - f (0,1,0) \\
    & = \pi ^4 (\pi ^2 + 1) /2 + \sin (e ^{\pi ^2}) + e ^{- \pi ^2} - \sin (1) - 1 
\end{align*}


\begin{exercise}
    Show that the vector field 
    \[     F(x,y) = \left\langle \frac{-y}{x^2 + y^2 } , \frac{x}{x^2 + y^2 }       \right\rangle     \]
    is not conservative, even though it has zero curl.
\end{exercise}

Let $C \subset \mathbb{R}^2 $ be the unit circle oriented counterclockwise. Take the parametrization $\gamma (t) = (\cos t , \sin t)$ with $0 \leq t \leq 2 \pi $. Then 
\begin{align*}
    F(\gamma (t)) & = \left\langle  \frac{- \sin t}{ \cos ^2 t + \sin ^2 t } ,  \frac{\cos t}{ \cos ^2 t + \sin ^2 t }  \right\rangle \\
    & = \langle - \sin t, \cos t \rangle
\end{align*}
Also, 
\[     \gamma ' (t) = \langle - \sin t , \cos t \rangle                   \]
Therefore, 
\begin{align*}
     \int_{C} F \cdot ds & = \int_0 ^{2 \pi } F(\gamma (t)) \cdot \gamma ' (t)  \, dt  \\ 
        & = \int_0 ^{2 \pi } (\sin ^2 t + \cos ^2 t ) \, dt  \\ 
     & = \int_0 ^{2 \pi } 1 \, dt \\
     & = 2 \pi \neq 0
\end{align*}
If $F$ was conservative, we would have $F = \nabla f$ for some scalar function $f (x,y)$. By the Fundamental Theorem of Calculus, we would have
\begin{align*}
   \int_C F \cdot ds & = f(\gamma (2 \pi )) - f (\gamma (0)) \\
    &= f (1,0) - f (1,0) \\
    & = 0                            
\end{align*}







\end{document}
