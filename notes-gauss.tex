\documentclass{ximera}

\title{Divergence Theorem}
\license{CC: 0}         % replace with an appropriate license, or set it in xmPreamble

\begin{document}

\begin{abstract}
    Divergence Theorem
\end{abstract}

\section{Divergence Theorem}

Let $F (x,y, z) $ be a space vector field, and $p $ a point in the domain of $F$. Let $\Sigma \subset \mathbb{R}^3 $ be a very small sphere around $p$, positively oriented (outwards).  Then the integral
\[       \iint_{\Sigma} F \cdot ds    \]
measures how much $F$ is pushing things away from $p$. It turns out we have another quantity that measures precisely that: 
\[       \text{div}(F)(p)  .    \]
Overall, if we have a region $E \subset \mathbb{R}^3$ with boundary $\partial E$ oriented positively, the two integrals
\begin{gather*}
      \iint_{\partial E}    F \cdot ds                  \\
      \iiint_{E}  \text{div}(F)  \, dA      
\end{gather*}
measure how much the vector field is pointing outside of $E$. The Divergence Theorem (aka Gauss Theorem) states that they agree.

\begin{theorem}[Divergence Theorem]  
Let $F(x,y, z)$ be a space vector field, and $E \subset \mathbb{R}^3$ a region inside the domain of $F$. Then
\[          \iint_{\partial E}    F \cdot ds  =    \iiint_{E}  \text{div}(F)  \, dV                  \]
\end{theorem}





\begin{exercise}

\end{exercise}








\end{document}

