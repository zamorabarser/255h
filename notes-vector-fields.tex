\documentclass{ximera}

\title{Vector fields, curl, and divergence}
\license{CC: 0}         % replace with an appropriate license, or set it in xmPreamble

\begin{document}

\begin{abstract}
    Vector fields, curl, and divergence
\end{abstract}

\section{Vector fields}

A planar vector field is a function $F : D \to \mathbb{R}^2$ with $D \subset \mathbb{R}^2$ a region we call the domain of $F$. 


A space vector field is a function $F : D \to \mathbb{R}^3$ with $D \subset \mathbb{R}^3$ a region we call the domain of $F$. 

A vector field in general corresponds to having a vector at each point. These are really good to model:
\begin{itemize}
    \item Force fields.
    \item Electromagnetic fields.
    \item Motions of fluids and wind.
    \item Ordinary differential equations.
\end{itemize}

You can draw vector fields on Desmos:

\href{https://www.desmos.com/calculator/eijhparfmd}{https://www.desmos.com/calculator/eijhparfmd}


\section{Flows}

Given a planar vector field $F: D \to \mathbb{R}^2$, a flow line is a differentiable curve $\gamma : [ a,b] \to D $ with 
\[      \gamma ' (t) = F (\gamma (t) )                        \]
for all $t$. This means that $\gamma $ models a particle that at each time, its velocity is the vector provided by $F$ at its position.

You can create animations of flows of planar vector fields using the following:
\begin{itemize}
    \item Clic on the following link: \\
    \href{https://drive.google.com/file/d/1e6GniqFvsR_vx6HhRw0y3NhjV37WHVMF/view?usp=sharing}{https://drive.google.com/file/d/1e6GniqFvsR\_vx6HhRw0y3NhjV37WHVMF/view?usp=sharing}

    Copy that code in your cliboard
    
    \item Go to the following website: 

    \href{https://animg.app/playground}{https://animg.app/playground}

    \item  Paste the code in the text box and render (can take a couple of minutes).

    \item Edit the vector fields on lines 24-28 and render again. 

    \item The free version of Animg only has three free renders per day, so do this with your friends.

\end{itemize}

Flows of space vector fields are defined analogously.

\begin{theorem}
    Let $F : D \to \mathbb{R}^2$ be a smooth vector field. Then for any point $p \in D$ there is a unique flow line of $F$ that starts at $p$. The same is true for space vector fields.
\end{theorem}


\section{Curl}


The curl of a vector field, measures how much rotation (swirl) is generated by its flow. This is measured in very distinct ways in the plane and in the space because angular momentum is encoded in the plane by a scalar and in the space by a vector.

\subsection{2D curl}

In the plane, rotation is simply either clockwise or counterclockwise.

\begin{definition}
The curl of a planar vector field 
\[   F(x,y) = \langle P (x,y) , Q (x,y) \rangle     \]  
is given by
\[      \text{curl}(F)  =  \frac{\partial Q}{\partial x}  - \frac{\partial P }{\partial y }                             \]
\end{definition}



\begin{itemize}
    \item The curl of a vector field $F$ at a point $p$ is positive if its flow generates counterclockwise swirl near $p$.
    \item The curl of a vector field $F$ at a point $p$ is negative if its flow generates clockwise swirl near $p$.
\end{itemize}

\begin{example}
    The vector field $F(x,y) = \langle -y / 2 , x / 2 \rangle $ generates a strong counterclockwise swirl and its curl is given by
    \[   \text{curl}(F) = 1    \]
\end{example}

\begin{example}
    The vector field $F(x,y) = \langle  x / 2 , y / 2 \rangle $ generates a strong flow, but does not generate any rotation. Hence its curl is 
    \[   \text{curl}(F) = 0     \]
\end{example}

\subsection{3D curl}

In the space, rotation (angular momentum) is encoded by a vector. The angular momentum of a rotation is a vector $v$ such that:
\begin{itemize}
    \item the direction of $v$ is the axis of rotation.
    \item the length of $v$ is the speed of rotation.
    \item looking back at the object from the tip of $v$, we see it spinning counterclockwise.
\end{itemize}

\begin{definition}
The curl of a space vector field 
\[   F(x,y, z) = \langle P (x,y, z) , Q (x,y, z) , R (x,y,z) \rangle     \]  
is given by
\[      \text{Curl}(F)  = \Big\langle     \frac{\partial R}{\partial y}  - \frac{\partial Q }{\partial z } ,    \frac{\partial P}{\partial z}  - \frac{\partial R }{\partial x } ,    \frac{\partial Q}{\partial x}  - \frac{\partial P }{\partial y }                         \Big\rangle     \]
    
\end{definition}

This formula is somewhat hard to remember, but if we consider the abstract vector 
\[ \nabla = \Big\langle \frac{\partial}{\partial x} , \frac{\partial}{\partial y} , \frac{\partial}{\partial z}  \Big\rangle ,  \] 
then
\[       \text{Curl}(F)  =  \nabla \times F       \]
becomes just a cross product.


\begin{example}
    The vector field $F(x,y, z) = \langle -y / 2 , x / 2 , 0\rangle $ generates a strong swirl around the $z$-axis. Its curl is given by 
    \[   \text{Curl}(F) = \langle 0 , 0 , 1 \rangle     \]
\end{example}


\begin{example}
    The vector field $F(x,y, z) = \langle  0, -z / 2 , y / 2 \rangle $ generates a strong swirl around the $x$-axis. Its curl is given by 
    \[   \text{Curl}(F) = \langle 1 , 0 , 0 \rangle     \]
\end{example}


\begin{example}
    The vector field $F(x,y, z) = \langle  x / 2 , y / 2 , z / 2  \rangle $ generates a strong flow, but does not generate any rotation. Hence its curl is 
    \[   \text{Curl}(F) =  \langle 0 , 0 ,  0  \rangle    \]
\end{example}


\section{Divergence}



The divergence of a vector field at a point measures how much the vector field is pointing away from that point. 
\begin{definition}
The divergence of a planar vector field
\[   F(x,y) = \langle P (x,y) , Q (x,y) \rangle     \]  
is given by
\[      \text{div}(F)  =  \frac{\partial P}{\partial x}  + \frac{\partial Q }{\partial y }        \]     
\end{definition}
\begin{definition}
The divergence of a space vector field
\[   F(x,y, z) = \langle P (x,y,z) , Q (x,y,z), R(x,y,z) \rangle     \]  
is given by
\[      \text{div}(F)  =  \frac{\partial P}{\partial x}  + \frac{\partial Q }{\partial y }  + \frac{\partial R}{\partial z }      \]     
\end{definition}

Even though the divergence formula is quite simple, it can be simplified more as as dot product:
\[    \text{div} (F) = \nabla \cdot   F  \]


\begin{example}
    The vector field $F(x,y) = \langle -y / 2 , x / 2 \rangle $ generates a strong counterclockwise swirl, but the flow lines are just rotating around. Its divergence is 
    \[   \text{div}(F) = 1    \]
\end{example}

\begin{example}
    The vector field $F(x,y) = \langle  x / 2 , y / 2 \rangle $ generates a strong flow that is blowing-up away from each point. Its divergence is 
    \[   \text{div}(F) = 1     \]
\end{example}

\section{Conservative vector fields}

\begin{definition}
    A vector field $F : D \to \mathbb{R}^3$ is conservative if it is the gradient of a function $f : D \to \mathbb{R}$. In such a case, the function $f$ is called a potential of $F$.
\end{definition}


\begin{theorem}
    A smooth conservative vector field $F$ has no curl. 
\end{theorem}

\begin{proof}
    In the plane,  a conservative vector field is 
    \[  F(x,y) =  \Big\langle \frac{\partial f}{\partial x} , \frac{\partial f}{\partial y}   \Big\rangle      \]
    Then its curl is
    \begin{align*}
        \text{curl}(F) & = \frac{\partial }{\partial x} \frac{\partial f}{\partial y} - \frac{\partial }{\partial y} \frac{\partial f}{\partial x}  \\
        & = \frac{\partial ^2 f }{\partial x \partial y } - \frac{\partial ^2 f }{\partial y \partial x }  \\
        & = 0 
    \end{align*}
    In the space,  a conservative vector field is 
    \[  F(x,y, z) =  \Big\langle \frac{\partial f}{\partial x} , \frac{\partial f}{\partial y}  , \frac{\partial f}{\partial z } \Big\rangle      \]
    Then its curl is
    \begin{align*}
        \text{Curl}(F) & = \Big \langle \frac{\partial }{\partial y} \frac{\partial f}{\partial z} - \frac{\partial }{\partial z} \frac{\partial f}{\partial y}, \frac{\partial }{\partial z} \frac{\partial f}{\partial x} - \frac{\partial }{\partial x} \frac{\partial f}{\partial z},   \frac{\partial }{\partial x} \frac{\partial f}{\partial y} - \frac{\partial }{\partial y} \frac{\partial f}{\partial x}  \Big \rangle  \\
        & =  \Big \langle  \frac{\partial ^2 f}{\partial y\partial z} - \frac{\partial ^2 f}{\partial z\partial y},  \frac{\partial ^2 f}{\partial z \partial x} - \frac{\partial ^2 f}{\partial x \partial z},   \frac{\partial ^2 f}{ \partial x \partial y} -  \frac{\partial ^2 f}{\partial y \partial x}  \Big \rangle  \\
        & = \langle 0,0,0 \rangle
    \end{align*}


\end{proof}

\begin{remark}
 There are non-conservative vector fields with no curl. For example,
 \[    F(x,y) = \Big\langle  \frac{-y}{x^2 + y^2 } , \frac{x}{x^2 + y^2}    \Big\rangle   \]
 has zero curl, but it is not conservative. The main issue is that $(0,0)$ is not in the domain of $F$.
\end{remark}
 
 \begin{theorem}
     If $F : D \to \mathbb{R}^2 $ is a planar smooth vector field with 
     \[     \text{curl} (F) = 0    \]
     and $D$ has no holes, then $F$ is conservative. 
 \end{theorem}

Technically, ``having no holes'' can be defined in terms of curves: any closed curve in $D$ can be continuously deformed within $D$ to a single point within $D$. For example, the domain of 
 \[    F(x,y) = \Big\langle  \frac{-y}{x^2 + y^2 } , \frac{x}{x^2 + y^2}    \Big\rangle   \] 
 is $\mathbb{R}^2 \backslash \{ (0,0) \}$, which contains the unit circle, a curve that cannot be continuously deformed to a single point within $D$. It is like having a rubber band stuck around a pole.


\section{Exercises on vector fields}

\begin{exercise}
    Consider the planar vector field 
    \[   F ( x , y ) = \langle e^x \cos y , \sin y + x^2 \rangle           \]
    Find its curl, divergence, and determine whether it is conservative or not. If it is conservative, find a potential function. 
\end{exercise}


The curl can be computed as
\begin{align*}
    \text{curl} (F) & = \frac{\partial}{\partial x} (  \sin y + x^2 ) - \frac{\partial }{\partial y } (e^x \cos y) \\
    & = 2x + e^x \sin y
\end{align*}
For the divergence, 
\begin{align*}
    \text{div} (F) & = \frac{\partial}{\partial x} (  e^x \cos y  ) + \frac{\partial }{\partial y } ( \sin y + x^2 ) \\
    & = e^x \cos y + \cos y 
\end{align*}
Since the curl is not zero, $F$ is not conservative.


\begin{exercise}
    Consider the planar vector field 
    \[   F ( x , y ) = \langle  y^2 - x^2  , 2xy  \rangle           \]
    Find its curl, divergence, and determine whether it is conservative or not. If it is conservative, find a potential function. 
\end{exercise}


The curl can be computed as
\begin{align*}
    \text{curl} (F) & = \frac{\partial}{\partial x} (  2xy ) - \frac{\partial }{\partial y } ( y^2  - x^2 ) \\
    & = 2y - 2y \\
    & = 0
\end{align*}
For the divergence, 
\begin{align*}
    \text{div} (F) & = \frac{\partial}{\partial x} ( y^2 - x^2 ) + \frac{\partial }{\partial y } ( 2xy ) \\
    & = - 2x + 2x \\
    & = 0
\end{align*}
Since the curl is zero, and  $F$ is defined everywhere, it is  conservative. To find the potential function, we perform some   ``partial integration'':
\begin{align*}
    \frac{\partial f}{\partial x} & = y^2 - x^2 \\
    \frac{\partial f}{\partial x} & = 2xy
\end{align*}
We get
\begin{align*}
    f (x,y) & = xy^2 - x^3/3 + a(y)\\
    f (x,y) & 
    = xy^2 + b(x)
\end{align*}
Matching terms, we get
\[     f(x,y)  = xy^2 - x^3/3  + C  \]


\begin{exercise}
    Consider the space vector field 
    \[   F ( x , y , z ) = \Big\langle  \frac{-y}{x^2 + y^2 } ,  \frac{x}{x^2 + y^2 } , z  \Big\rangle           \]
    Find its curl, divergence, and determine whether it is conservative or not. If it is conservative, find a potential function. 
\end{exercise}

Before we compute the curl and divergence, we find the partial derivatives
\begin{align*}
    \frac{\partial P}{\partial x} & = \frac{2xy}{(x^2 + y^2 )^2} \\
    \frac{\partial P}{\partial y} & =  \frac{y^2 - x^2}{x^2 + y^2 }  \\
    \frac{\partial P}{\partial z} & = 0 \\
    \frac{\partial Q}{\partial x} & = \frac{y^2 - x^2}{x^2 + y^2 } \\
    \frac{\partial Q}{\partial y} & = \frac{- 2xy}{(x^2 + y^2 )^2}   \\
    \frac{\partial Q}{\partial z} & = 0 \\
    \frac{\partial R}{\partial x} & = 0 \\
    \frac{\partial R}{\partial y} & = 0 \\
    \frac{\partial R}{\partial z} & = 1 \\
\end{align*}


The curl is
\begin{align*}
      \text{Curl} (F)     & =  \Big\langle  0 - 0 , 0 - 0,  \frac{y^2 - x^2}{x^2 + y^2 } -  \frac{y^2 - x^2}{x^2 + y^2 }    \Big\rangle      \\
    & = \langle 0,0,0 \rangle 
\end{align*}
The divergence is 
\begin{align*}
    \text{div} (F) & = \frac{2xy}{(x^2 + y^2 )^2} - \frac{2xy}{(x^2 + y^2 )^2} + 1 \\
    & = 1 
\end{align*}
The curl is zero, but $F$ is not defined along the $z$-axis, where $x^2 + y^2 = 0$. We will see later that $F$ is not conservative. 





\end{document}
