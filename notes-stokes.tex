\documentclass{ximera}

\title{Stokes Theorem}
\license{CC: 0}         % replace with an appropriate license, or set it in xmPreamble

\begin{document}

\begin{abstract}
    Stokes Theorem
\end{abstract}

\section{Stokes Theorem}

Let $F (x,y, z) $ be a space vector field,  $p $ a point in the domain of $F$, and $\rm{v}$ a unit vector based at $p$. Let $C\subset \mathbb{R}^2 $ be a very small circle centered at $p$, around $\rm{v}$,  which travels counterclockwise when seen from the tip of $\rm{v}$. Then the integral
\[       \int_C F \cdot  d \gamma     \]
measures how much rotation is generated by $F$ around the axis $\rm{v}$.  It turns out we have another quantity that measures precisely that: 
\[       \text{Curl}(F)(p) \cdot \rm{v} .    \]


Overall, if we have an oriented surface $ \Sigma \subset \mathbb{R}^3$ with boundary $\partial \Sigma $ oriented counterclockwise, the two integrals
\begin{gather*}
      \int_{\partial \Sigma }    F \cdot  d \gamma                   \\
      \iint_{\Sigma }  ( \text{Curl}(F) \cdot  N    )\,  dS      
\end{gather*}
measure how much rotation the vector field $F$ generates over $\Sigma$ (with axis perpendicular to the surface). Stokes Theorem  states that they agree.

\begin{theorem}[Stokes Theorem]  
Let $F(x,y, z)$ be a space vector field, and $ \Sigma  \subset \mathbb{R}^3$ an oriented surface with boundary $\partial \Sigma$, oriented counterclockwise, when seen from the tip of $N$. Then
\[          \int_{\partial \Sigma}    F \cdot  d \gamma   =    \iint_{\Sigma }  \text{Curl}(F)  \cdot dS                  \]
\end{theorem}

\begin{exercise}
    Let $C \subset \mathbb{R}^3$ be the circle $x^2 + y^2 = 25$ in the $xy$-plane, oriented counterclockwise, when seen from above. Compute
    \[    \int_C    \langle  x^2 - 2z - 5 , 3x - e^y  , z^3 - xy + 1  \rangle \cdot  d \gamma       \]
\end{exercise}
Set $F  =   \langle  x^2 - 2z - 5 , 3x - e^y  , z^3 - xy + 1  \rangle $ and compute
\[    \text{Curl} (F) = \langle - x   , -2 + y , 3 \rangle                   \]
If we define $\Sigma$ to be disk $x^2 + y^2 \leq 25 $ in the $xy$-plane, oriented upward, we have $\partial \Sigma = C$, and by Stokes Theorem, 
\begin{align*}
     \int_C    \langle & x^2 - 2z - 5 , 3x - e^y  , z^3 - xy + 1  \rangle \cdot  d \gamma  \\
     & = \iint_{\Sigma}  \langle - x   , -2 + y , 3 \rangle   \cdot dS \\
     & = \iint_{\Sigma}  \langle - x   , -2 + y , 3 \rangle   \cdot  \langle 0,0,1 \rangle \,  dS \\
     & = \iint_{\Sigma}   3  \,  dS  \\
     & = 75 \pi 
\end{align*}



\begin{exercise}
    Let $C\subset \mathbb{R}^3$ be the curve that travels along straight lines first from $(1,0,0)$ to $(0,0,1)$, then from $(0,0,1)$ to $(0,1,0)$, and then from $(0,1,0)$ to $(1,0,0)$. Compute
    \[    \int_C \langle   3x + y,  e^y - 2x , 5 z - 4x - 1    \rangle \cdot  d \gamma            \]
\end{exercise}
Set $F  =   \langle   3x + y,  e^y - 2x , 5 z - 4x - 1    \rangle $ and compute
\[    \text{Curl} (F) = \langle  0, 4 , -3  \rangle                   \]
If we define $\Sigma$ to be the triangle with vertices $(1,0,0)$, $(0,1,0)$, $(0,0,1)$, oriented upward, by Stokes Theorem we have 
\[       \int_C \langle   3x + y,  e^y - 2x , 5 z - 4x - 1    \rangle \cdot  d \gamma       = - \iint_{\Sigma} \langle 0,4,-3\rangle \cdot dS             \]
where the minus sign appears because when we look at $C$ from the tip of $N = \langle 1,1,1\rangle$, it goes clockwise. Note that $\langle 0,4,-3 \rangle \cdot N  = 4-3 = 1$, so 
\[     \iint_{\Sigma} \langle 0,4,-3\rangle \cdot dS        =   \iint_{\Sigma} 1 \, dS = \text{Area}(\Sigma)   \]
The triangle is equilateral with side $\sqrt{2}$, so its area is $\frac{1}{2}$, and 
\[ \int_C \langle   3x + y,  e^y - 2x , 5 z - 4x - 1    \rangle \cdot  d \gamma       = - \frac{1}{2} \]



\begin{exercise}
    Let $C\subset \mathbb{R}^3$ be the curve that goes from $(1,0)$ to $(-1,0)$ along the arc $x^2 + y^2 = 1$, $y \geq 0$, in the $xy$-plane, followed by the curve that goes back to $(1,0)$ along the arc $x^2 + z^2 = 1$, $z\geq 0$, in the $xz$-plane. Compute
    \[    \int_C \langle    3x + 2 + z^2 , 4 \cos y - z^2  , 2y - 5  \rangle \cdot  d \gamma            \]
\end{exercise}
Set $F = \langle    3x + 2 + z^2 , 4 \cos y   , 2y - 5  \rangle$. We compute
\[  \text{Curl} (F) = \langle   2 , 2z , 0   \rangle          \]
If we define $\Sigma $ to be the portion of the sphere $x^2 + y^2 +  z^2 = 1$ with $y \geq 0$, $z \geq 0$, oriented upward, we get $\partial \Sigma = C$. Then by Stokes Theorem, 
\[      \int_C \langle    3x + 2 + z^2  , 4 \cos y  , 2y - 5  \rangle \cdot  d \gamma     = \iint_{\Sigma} \langle   2 ,  2z , 0   \rangle \cdot dS               \]
Parametrizing $\Sigma$, we use $\varphi : [0, \pi] \times [0, \pi / 2] \to \Sigma$ with
\[     \varphi (u,v)  = (   \cos u \sin v  , \sin u \sin v , \cos v   )       , \]
then
\begin{align*}
    \frac{\partial \varphi}{\partial u } & = \langle -\sin u \sin v , \cos u \sin v , 0     \rangle \\
    \frac{\partial \varphi}{\partial v } & = \langle \cos u \cos v , \sin u \cos v , - \sin v     \rangle \\
    \frac{\partial \varphi}{\partial u } \times \frac{\partial \varphi}{\partial v }  & = \langle - \cos u \sin ^2 v  , - \sin u \sin ^2 v  , - \sin v \cos v   \rangle  
\end{align*}
Note that $    \frac{\partial \varphi}{\partial u } \times \frac{\partial \varphi}{\partial v } $ points inwards, so $\varphi$ is NOT compatible with the orientation. The integrand becomes
\begin{align*}
      \langle 2 ,   2 \cos v ,0 \rangle & \cdot               \langle - \cos u \sin ^2 v  , - \sin u \sin ^2 v  , - \sin v \cos v   \rangle   \\
      & = -2 \cos u \sin ^2 v  -  2 \sin u \cos v \sin ^2 v 
\end{align*}
Then (recalling that $\varphi$ was not compatible with the orientation), we conclude
\begin{align*}
    \iint_{\Sigma} \langle   2 ,  2z , 0   \rangle \cdot dS  & = -  \int_{0 }^{\pi} \int_0 ^{\pi/2} [-2 \cos u \sin ^2 v  - 2 \sin u \cos v \sin ^2 v   ] \, d v du \\
    &  =  0  +   4 \int_0 ^{ \pi / 2 }  \cos v \sin ^2 v \, dv   \\
    & =   4 \left[  \frac{\sin ^3 v}{3 }   \right] _{v = 0} ^{ \pi /2} \\
    & =  \frac{4}{3} 
\end{align*}

























\end{document}

