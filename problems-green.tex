\documentclass{ximera}

\title{Green's Theorem}
\license{CC: 0}         % replace with an appropriate license, or set it in xmPreamble

\begin{document}

\begin{abstract}
    Green's Theorem
\end{abstract}


\section{Green's Theorem}
\begin{exercise}
    Let $C \subset \mathbb{R}^2$ be the curve that travels along straight lines from $(0,0)$ to $(4,0)$,  then from $(4,0) $ to $(0,3)$, and then from $(0,3)$ back to $(0,0)$. Compute
    \[
        \int_C \langle  3x-y^2  - 1 , x + 2y \rangle \cdot  d \gamma 
    \]
\end{exercise}

\begin{exercise}
    Let $C \subset \mathbb{R}^2$ be the curve that travels along straight lines from $(-1,2)$ to $(1,2)$,  then from $(1,2) $ to $(1,3)$, then from $(1,3) $ to $(-1,3)$, and then from $(-1,3)$ back to $(-1,2)$. Compute
    \[
        \int_C \langle x^2 y + 3x - 5, 2xy + e^y + 3 \rangle \cdot  d \gamma 
    \]
\end{exercise}

\begin{exercise}
    Let $C \subset \mathbb{R}^2$ be the circle $x^2 + y^2 = 9$ oriented counterclockwise. Compute
    \[
        \int_C \langle xy + 5y^3 + 2 , x - y e ^y  \rangle \cdot  d \gamma 
    \]
\end{exercise}

\begin{exercise}
    Let $C \subset \mathbb{R}^2$ be the ellipse $\frac{x^2}{4} + \frac{y^2}{9} = 1$ oriented counterclockwise. Compute
    \[
        \int_C \langle -3y,\; 2x + y \rangle \cdot  d \gamma 
    \]
\end{exercise}

\begin{exercise}
    Let $C \subset \mathbb{R}^2$ be the curve that travels along straight lines from $(-2,4)$ to $(-2,0)$, then from $(-2,0)$ to $(2,0)$, then from $(2,0)$ to $(2,4)$, and then back to $(-2,4)$ along the parabola $y = x^2$.  Compute
    \[
        \int_C \langle 3 y + x^2 - 2,  2x + y + 7 \rangle \cdot  d \gamma 
    \]
\end{exercise}
\begin{exercise}
    Let $C \subset \mathbb{R}^2$ be the curve that travels along straight lines from $(0,0)$ to $(3,1)$,  then from $(3,1) $ to $(4,4)$, then from $(4,4) $ to $(1,3)$, and then from $(1,3)$ back to $(0,0)$. Compute
    \[
        \int_C \langle y - 3 x + 2,  x^2 - 3 y + 5 \rangle \cdot  d \gamma 
    \]
\end{exercise}




\begin{exercise}
    Let $C_1 \subset \mathbb{R}^2$ be the curve that travels along straight lines from $(0,0)$ to $(3,0)$,  then from $(3,0) $ to $(4,4)$. Let $C_2 \subset \mathbb{R}^2$ be the curve that travels along straight lines from $(0,0)$ to $(0,5)$,  then from $(0,5) $ to $(4,4)$. Let $C_3 \subset \mathbb{R}^2$ be the straight line from $(0,0)$ to $(4,4)$. Let $F : \mathbb{R}^2 \to \mathbb{R}^2$ be a vector field with positive curl. Among the integrals
    \[    \int_{C_1} F\cdot  d \gamma_1  , \,\,\,\,\,\,\,\,\,\,\,\,\,\,\,\,\,   \int_{C_2} F\cdot  d \gamma_2  , \,\,\,\,\,\,\,\,\, \,\,\,\,\,\,\,\,  \int_{C_3} F\cdot  d \gamma_3  ,        \]
    which one is the largest and which one the smallest?
\end{exercise}





















\end{document}
