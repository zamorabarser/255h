\documentclass{ximera}

\title{Stokes Theorem}
\license{CC: 0}         % replace with an appropriate license, or set it in xmPreamble

\begin{document}

\begin{abstract}
    Stokes Theorem
\end{abstract}


\section{Stokes Theorem}



\begin{exercise}
Let $C\subset \mathbb{R}^3$ be the intersection of the cylinder  $x^2 + y^2 = 16$ and plane $z=2$, oriented counterclockwise when viewed from above. Compute
\[  \int_C \langle -y, x,  2z\rangle \cdot  d \gamma   \]
\end{exercise}

\begin{exercise}
Let $C\subset \mathbb{R}^3$ be the trajectory that travels along straight lines through the points $(0,1,0)$, $(4,1,0)$, $(4,1,2)$, $(0,1,2)$, and back to $(0,1,0)$. Compute
\[   \int_C \langle y^2 + z,  x y,  x - z\rangle \cdot  d \gamma   \]
\end{exercise}


\begin{exercise}
Let $C\subset \mathbb{R}^3$ be the intersection of the cylinder  $y^2 + z^2 = 16$ and plane $x=5$, oriented clockwise when seen from the tip of the $x$-axis. Compute
\[
\int_C \langle 5x - 3,  1 -  z,  2y + 4   \rangle \cdot  d \gamma 
\]
\end{exercise}


\begin{exercise}
Let $C\subset \mathbb{R}^3$ be the curve that travels along straight lines first from $(3,0,0)$ to $(0,5,0)$, then from $(0,5,0)$ to $(0,0,15)$, and then from $(0,0,15)$ to $(3,0,0)$. Compute
\[   \int_C \langle 2y,   -x + z,   y   \rangle \cdot  d \gamma    \]
\end{exercise}



\begin{exercise}
    Let $C\subset \mathbb{R}^3$ be the curve that goes from $(2,0)$ to $(-2,0)$ along the arc $x^2 + y^2 = 4$, $y \geq 0$, in the $xy$-plane, followed by the curve that goes back to $(2,0)$ along the parabola $z = x^2 - 4 $  in the $xz$-plane. Compute
    \[    \int_C \langle    x + y , z + x , y + z   \rangle \cdot  d \gamma            \]
\end{exercise}



\begin{exercise}
    Let $C_1\subset \mathbb{R}^3$ be the circle $x^2 + y^2 = 1$ in the $xy$-plane, oriented counterclockwise, and $C_2\subset \mathbb{R}^3$ the circle $x^2 + y^2 = 1$ in the plane $z = 3$, oriented counterclockwise. Let $F: \mathbb{R}^3 \to \mathbb{R}^3$ be a vector field with 
    \[      \text{Curl}(F) (x,y,z) = \langle x, y, 0 \rangle  .  \]
    Which one is larger?
    \[     \int_{C_1} F \cdot  d \gamma _1 \,\,\,\,\,\,\,\,\,\,\, \text{ or }    \,\,\,\,\,\,\,\,\,\,\,          \int_{C_2} F \cdot  d \gamma_2       \]
\end{exercise}




\begin{exercise}
    Let $C_1\subset \mathbb{R}^3$ be the circle $x^2 + y^2 = 1$ in the $xy$-plane, oriented counterclockwise, and $C_2\subset \mathbb{R}^3$ the circle $x^2 + y^2 = 16$ in the $xy$-plane, oriented counterclockwise. Let $F: \mathbb{R}^3 \to \mathbb{R}^3$ be a vector field with 
    \[      \text{Curl}(F) (x,y,z) = \langle 0, 0, 1 \rangle  .  \]
    Which one is larger?
    \[     \int_{C_1} F \cdot  d \gamma_1  \,\,\,\,\,\,\,\,\,\,\, \text{ or }    \,\,\,\,\,\,\,\,\,\,\,          \int_{C_2} F \cdot  d \gamma _2      \]
\end{exercise}




\begin{exercise}
    Let $C_1\subset \mathbb{R}^3$ be the curve that travels along straight lines from $(4,0,0)$, to $(0,4,0)$, from $(0,4,0)$ to $(0,0,4)$, and from $(0,0,4) $ to $(4,0,0)$.  Let  $C_2\subset \mathbb{R}^3$ be the curve that travels along straight lines from $(1,0,0)$, to $(0,1,0)$, from $(0,1,0)$ to $(0,0,1)$, and from $(0,0,1) $ to $(1,0,0)$.  Let $F: \mathbb{R}^3 \to \mathbb{R}^3$ be a vector field with 
    \[      \text{Curl}(F) (x,y,z) = \langle x, y, z \rangle  .  \]
    Which one is larger?
    \[     \int_{C_1} F \cdot  d \gamma _1 \,\,\,\,\,\,\,\,\,\,\, \text{ or }    \,\,\,\,\,\,\,\,\,\,\,          \int_{C_2} F \cdot  d \gamma _2     \]
\end{exercise}




































\end{document}
