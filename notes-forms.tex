\documentclass{ximera}

\title{Forms}
\license{CC: 0}         % replace with an appropriate license, or set it in xmPreamble

\begin{document}

\begin{abstract}
    Forms
\end{abstract}

\section{Determinant-like functions}

After all, the Fundamental Theorem of Calculus, the Curved Fundamental Theorem of Calculus, Green's Theorem, Stokes Theorem, and the Divergence Theorem, are instances of the Generalized Stokes Theorem. 

To state it, we need to talk about forms, and to talk about forms, we need determinant-like functions called $k$-forms. Let's recall a couple of properties of determinants:

\begin{proposition}
    The determinants satisfy the following properties:
    \begin{itemize}
        \item (multilinearity) If all but one row is fixed, the determinant depends linearly on the non-fixed row.

        \textbf{Examples:}
            \[   \det \begin{pmatrix}
                a_1 & a_2 & a_3 \\
                b_1 & b_2 & b_3 \\
                c_1 & c_2 & c_3
            \end{pmatrix}      + \, \det \begin{pmatrix}
                a_1 & a_2 & a_3 \\
                b_1 & b_2 & b_3 \\
                d_1 & d_2 & d_3
            \end{pmatrix}      = \,  \det \begin{pmatrix}
                a_1 & a_2 & a_3 \\
                b_1 & b_2 & b_3 \\
                c_1 + d_1 & c_2 + d_2 & c_3 + d_3
            \end{pmatrix}             \]
\[   \det \begin{pmatrix}
                 5 &  8 \\
                 1  &  2
            \end{pmatrix}  + \,  \det \begin{pmatrix}
                 5 &  8 \\
                 2  &  2 
            \end{pmatrix} = \,  \det \begin{pmatrix}
                 5 &  8 \\
                 3  &  4
            \end{pmatrix}
            \]
    \[   \det \begin{pmatrix}
                3 & -2 & 5 \\
                5 & -1 & 2 \\
                3 & 0 & -6
            \end{pmatrix}    
            + \, \det \begin{pmatrix}
                3 & -2 & 5 \\
                -2  & 3 & 4 \\
                3 & 0 & -6
            \end{pmatrix}      
            = \,  \det \begin{pmatrix}
                3 & -2 & 5 \\
                3 & 2 & 6 \\
                3 & 0 & -6
            \end{pmatrix} 
            \]
            
        \item (alternancy) If two rows are exchanged, the determinant changes sign.

        \textbf{Examples:}
            \[   \det \begin{pmatrix}
                a_1 & a_2 & a_3 \\
                b_1 & b_2 & b_3 \\
                c_1 & c_2 & c_3
            \end{pmatrix}        = -  \,  \det \begin{pmatrix}
                a_1 & a_2 & a_3 \\
                c_1 & c_2 & c_3 \\
                b_1 & b_2 & b_3 
            \end{pmatrix}              \]
            
            \[   \det \begin{pmatrix}
                -1 & 9 \\
                6 & 4 
            \end{pmatrix}  = -  \,  \det \begin{pmatrix}
                6 & 4  \\
                -1 & 9 
            \end{pmatrix}
            \]
            \[   \det \begin{pmatrix}
                5 & 2 & 7 \\
                4 & -2 &  3 \\
                -3 & 4 & 0
            \end{pmatrix}        = -  \,  \det 
            \begin{pmatrix}
                4 & -2 &  3 \\
                5 & 2 & 7 \\
                -3 & 4 & 0
            \end{pmatrix}
            \]
    
    \end{itemize}
\end{proposition}



\begin{definition}
 Let $k, n \in \mathbb{N}$ be natural numbers. A simple $k$-form in $\mathbb{R}^n$ is a continuous function $\omega : (\mathbb{R}^n)^k \to \mathbb{R}$ that takes $k$ vectors in $\mathbb{R}^n$ (also known as a $k \times n$ matrix) and returns a number, satisfying the following properties:
 \begin{itemize}
     \item (multilinearity) When all but one of the vectors are fixed, $\omega$ depends linearly on the non-fixed vector: 
     \[         \omega (\mathrm{v}_1,  \ldots , \mathrm{x} , \ldots  ,  \mathrm{v}_k  ) + \omega (\mathrm{v}_1,  \ldots , \mathrm{y} , \ldots  ,  \mathrm{v}_k  )  = \omega (\mathrm{v}_1,  \ldots , \mathrm{x} + \mathrm{y} , \ldots  ,  \mathrm{v}_k  )                     \]
     \item (alternancy) If two inputs are exchanged, $\omega$ changes sign:
     \[   \omega(\mathrm{v}_1 , \ldots, \mathrm{v}_i, \mathrm{v}_{i+1}, \ldots , \mathrm{v}_k) =  - \omega(\mathrm{v}_1 , \ldots, \mathrm{v}_{i+1}, \mathrm{v}_i, \ldots , \mathrm{v}_k)                         \]
 \end{itemize}
A simple $0$-form is just a number.

\end{definition}


It turns out that $k$-forms only exist if $k \leq n$ in the sense that if $k > n$, any simple $k$-form in $\mathbb{R}^n$ is just zero. 



\section{Mini-determinants}

The main examples of simple $k$-forms are mini-determinants. 

\begin{definition}
    Let $k,n \in \mathbb{N}$ be natural numbers, and $i_1 , \ldots , i_k  $ a collection of indices. The simple $k$-form (mini-determinant)
    \[   dx^{i_1} \wedge \ldots \wedge  dx^{i_k}  : (\mathbb{R}^n) ^k \to \mathbb{R}                             \]
    is defined as:
    \[      dx^{i_1} \wedge \ldots \wedge  dx^{i_k}  \begin{pmatrix}
        a_{11} & \ldots  & a_{1n} \\
        \vdots &  \ddots &  \ldots \\
        a_{k1} & \ldots & a_{kn}
    \end{pmatrix}    =  
    \det \begin{pmatrix}
        a_{1i_1} & \ldots  & a_{1i_n} \\
        \vdots &  \ddots &  \ldots \\
        a_{ki_1} & \ldots & a_{ki_n}
    \end{pmatrix}
    \]
\end{definition}

\begin{notation}
    In $\mathbb{R}^2$, $x^1 = x$ and $x^2 = y$. In $\mathbb{R}^3$, $x^1 = x$, $x^2 = y$, and $x^3 = z$.
\end{notation}

\textbf{Examples:}
\begin{align*}
     dx \wedge dy    \begin{pmatrix}
        3  & -3  & -2 \\
        5 & 2 & 7
    \end{pmatrix}  &  = 
    \det 
    \begin{pmatrix}
        3 & -3 \\
        5 & 2
    \end{pmatrix}\\ 
    & \\
    dx \wedge dz    \begin{pmatrix}
        3  & -3  & -2 \\
        5 & 2 & 7
    \end{pmatrix}   & = 
    \det 
    \begin{pmatrix}
        3 & -2 \\
        5 & 7
    \end{pmatrix}\\
    & \\
    dy \wedge dz    \begin{pmatrix}
        3  & -3  & -2 \\
        5 & 2 & 7
    \end{pmatrix}   & = 
    \det 
    \begin{pmatrix}
        -3 & -2 \\
        2 & 7
    \end{pmatrix}\\
    & \\
    dx \wedge dy    \begin{pmatrix}
        5  & -1   \\
        -3 & 4 
    \end{pmatrix}  &  = 
    \det 
    \begin{pmatrix}
        5  & -1   \\
        -3 & 4
    \end{pmatrix}
\end{align*}

\begin{align*}
    dx (4,-6,7) & = 4 \\
    & \\
    dy (4,-6,7) & = -6 \\
    & \\
    dz (4,-6,7) & = 7 \\
    & \\
    dx (13, -5) & = 13 \\
    & \\
    dy (13, -5) & = -5  
\end{align*}


\[    dx \wedge dy \wedge dz \begin{pmatrix}
    3 & -4  & 7 \\
    11 & 2 & -3 \\
    9 & 6 & -1
    \end{pmatrix}  = \det 
    \begin{pmatrix}
    3 & -4  & 7 \\
    11 & 2 & -3 \\
    9 & 6 & -1
    \end{pmatrix}         \]


\begin{align*}
    dx^1 \wedge dx^2   \begin{pmatrix}
     4  & 7 & 3 & -9 \\
    -3 & 11 & 6 & 5 
    \end{pmatrix}  & =  \det \begin{pmatrix}
     4  & 7  \\
    -3 & 11  
    \end{pmatrix}    \\
    & \\
    dx^2 \wedge dx^3   \begin{pmatrix}
     4  & 7 & 3 & -9 \\
    -3 & 11 & 6 & 5 
    \end{pmatrix}  & =  \det \begin{pmatrix}
     7  & 3  \\
    11  & 6  
    \end{pmatrix}    \\
    & \\
    dx^3 \wedge dx^4   \begin{pmatrix}
     4  & 7 & 3 & -9 \\
    -3 & 11 & 6 & 5 
    \end{pmatrix}  & =  \det \begin{pmatrix}
     3  & -9  \\
    6  & 5  
    \end{pmatrix}    
\end{align*}


\begin{theorem}
    All simple $k$-forms in $\mathbb{R}^n$ are linear combinations of mini-determinants. Moreover, the set 
    \[   \{ dx^{i_1} \wedge \ldots \wedge dx^{i_k} \, \vert \, 1 \leq i_1 < \ldots < i_k \leq n \}                        \]
    is a basis of the vector space of simple $k$-forms in $\mathbb{R}^n$.
\end{theorem}

\section{Differentiable forms}

\begin{definition}
    A smooth $k$-form with domain $U \subset \mathbb{R}^n$ is a function 
    \[   \omega : U \to \{ \text{simple } k\text{-forms in }\mathbb{R}^n \}       \]
    with smooth coefficients. That is, 
    \[      \omega (x) = \sum_{i_1 < \ldots < i_k} P_{i_1 , \ldots , i_k} (x) \, dx^{i_1} \wedge \ldots \wedge dx^{i_k} ,          \]
    with each function $P_{i_1 , \ldots , i_k}$ being smooth  (a smooth $0$-form is just a smooth function $f : U \to \mathbb{R}$).
\end{definition}


\textbf{Examples:} 

\begin{itemize}
    \item Smooth $1$-forms in $\mathbb{R}$:
    \begin{gather*}
        f(x) \, dx \\
        x^2 e ^x \, dx \\
        (3 + \cos x ) \, dx \\
        (5x - \sqrt{x} ) \, dx
    \end{gather*}

    \item Smooth $1$-forms in $\mathbb{R}^2$:
    \begin{gather*}
        P \, dx + Q \, dy \\
        3xy \, dx + x^2 \cos y \, dy \\
        5y^2 e^x \, dx \\
        (\sqrt{x} + \cos y ) \, dx + \frac{1}{x} \, dy
    \end{gather*}

    \item Smooth $1$-forms in $\mathbb{R}^3$:
    \begin{gather*}
        P \, dx + Q \, dy + R \, dz \\
        y^2 \cos x \,dx + x \sin z \, dy + z x^2 \, dz \\
        2xz \, dy + ze^y\cos x \,dz    \\
        (\sin y + \sin z ) \, dx +  e^x \cos y  \, dy + (\cos x - 1 ) \, dz
    \end{gather*}

    
    \item Smooth $2$-forms in $\mathbb{R}^2$:
    \begin{gather*}
        f(x,y) \, dx \wedge dy \\
        (e ^x \sin y + 3xy) \,   dx \wedge dy\\
         (5y - x \sqrt{y})\, dx \wedge dy    \\
         (e^x - x \cos y )\, dx \wedge dy
    \end{gather*}

    
    \item Smooth $2$-forms in $\mathbb{R}^3$:
    \begin{gather*}
        P \, dy \wedge dz + Q \, dz \wedge dx + R \, dx \wedge dy \\
        (5x^2 + e^z) \,   dx \wedge dy + (2x \sin z  + 3) \, dx \wedge dz \\
         ( 3 \sqrt{x} + \cos y )\, dx \wedge dy - z\, dy \wedge dz    \\
         (7y + y^2 e^x)\, dy \wedge dz + (2e^y - \cos x) \, dx \wedge dy
    \end{gather*}

    \item Smooth $3$-forms in $\mathbb{R}^3$:
    \begin{gather*}
        f(x,y,z) \, dx \wedge dy \wedge dz \\
        (e^x - y^2 z + \cos z) \, dx \wedge dy \wedge dz \\
        (3xyz + y^2 - \sqrt{z}) \, dx \wedge dy \wedge dz 
    \end{gather*}

\end{itemize}


General form of a 
\begin{center}
    \begin{tabular}{|c|c|c|c|}
    \hline
         &  in $\mathbb{R}$  & in $\mathbb{R}^2$ & in $\mathbb{R}^3$  \\
\hline
$0$-form   &  $f(x)$  & $f(x,y)$ & $f(x,y,z)$  \\
\hline
$1$-form   &  $f(x) \, dx$  & $P\,dx + Q \, dy$ & $P\, dx + Q \, dy + R \, dz$  \\
\hline
$2$-form   &  $0$  & $f(x,y)\,dx \wedge dy$ & $P\, dy \wedge dz + Q \, dz \wedge dx + R \, dx \wedge dy$  \\
\hline
$3$-form   &  $0$  & $0$ & $f(x,y,z ) \, dx \wedge dy \wedge dz $  \\
\hline
$4$-form   &  $0$  & $0$ & $0 $  \\
\hline
         
    \end{tabular}
\end{center}

\section{Manifolds}

In general, $k$-forms can be integrated over $k$-dimensional manifolds. $0$-dimensional manifolds are points, $1$-dimensional manifolds are curves, $2$-dimensional manifolds are surfaces.

\begin{definition}
    A parametrized $k$-manifold in $\mathbb{R}^n$ is a smooth map $\varphi : D \to \mathbb{R}^n$ with $D \subset \mathbb{R}^k$ a region. We often also call $M : = \varphi (D) \subset \mathbb{R}^n$ the manifold.  
\end{definition}

\begin{definition}
    The integral of a smooth $k$-form $\omega$ with domain  $U \subset \mathbb{R}^n$ over a parametrized $k$-manifold $\varphi : D \to M \subset U$ is given by 
    \[  \int \cdots \int_{M} \omega \, d \varphi  : =   \int \cdots \int _D \omega (\varphi(x_1, \ldots , x_k) ) \left(   \frac{\partial \varphi}{\partial x_1} , \ldots ,     \frac{\partial \varphi}{\partial x_k}    \right )  \, dx_1 \ldots dx_k                       \]
\end{definition}


\begin{example}[$k=1$, $n=1$] A differentiable $1$-form in $\mathbb{R}$ has the formula
\[          \omega  = f(x ) \, dx    .   \]
    Given an interval $[a,b] \subset \mathbb{R}$, the inclusion $ \varphi : [a,b] \to \mathbb{R}$ is a parametrized $1$-manifold.  Then
    \[
           \int_{[a,b]} \omega \, d \varphi    = \int_a ^b f(x)\, dx 
    \]
    This is the usual notion of integral.
\end{example}


\begin{example}[$k=1$, $n=2$] A differentiable $1$-form in $\mathbb{R}^2$ has the formula
\[          \omega  = P \, dx + Q \, dy    .   \]
    A parametrized $1$-manifold is a curve $\gamma : [a,b] \to C \subset  \mathbb{R}^2$. Then
    \begin{align*}
           \int_{C} \omega \, d \gamma  & = \int_a ^b [ P (\gamma (t)) \, dx (\gamma ' (t)) + Q (\gamma (t)) \, dy (\gamma ' (t))      ] \, dt  \\
           & = \int_a ^b \langle P (\gamma (t)) , Q (\gamma (t)) \rangle \cdot \gamma ' (t) \, dt \\
           & = \int _{C} \langle P, Q \rangle  \cdot d \gamma  
    \end{align*}
    This is the work of $\langle P, Q \rangle $ along $C$.
\end{example}





\begin{example}[$k=1$, $n=3$] A differentiable $1$-form in $\mathbb{R}^3$ has the formula
\[          \omega  = P \, dx + Q \, dy  + R \, dz   .   \]
    A parametrized $1$-manifold is a curve $\gamma : [a,b] \to C \subset  \mathbb{R}^3$. Then
    \begin{align*}
           \int_{C} \omega \, d \gamma  & = \int_a ^b [ P (\gamma (t)) \, dx (\gamma ' (t)) + Q (\gamma (t)) \, dy (\gamma ' (t))   + R(\gamma (t) ) \, dz (\gamma ' (t))    ] \, dt  \\
           & = \int_a ^b \langle P (\gamma (t)) , Q (\gamma (t)), R (\gamma (t)) \rangle \cdot \gamma ' (t)  \, dt \\
           & = \int _{C} \langle P, Q , R  \rangle  \cdot d \gamma 
    \end{align*}
    This is the work of $\langle P, Q, R \rangle $ along $C$.
\end{example}

\begin{example}[$k=2$, $n=2$] A differentiable $2$-form in $\mathbb{R}^2$ has the formula
\[          \omega  = f(x ,y) \, dx \wedge dy   .   \]
    Given a region  $D \subset \mathbb{R}^2$, the inclusion $ \varphi : D \to \mathbb{R}^2 $ is a parametrized $2$-manifold.  Then
    \[
           \iint_{D} \omega \, d \varphi    = \iint_D f(x,y)\, dA 
    \]
    This is the usual notion of integral.
\end{example}



\begin{example}[$k=2$, $n=3$] A differentiable $2$-form in $\mathbb{R}^3$ has the formula
\[          \omega  = P \, dy\wedge dz + Q \, dz \wedge dx + R \, dx \wedge dy   .   \]
    A parametrized $2$-manifold is a surface $\varphi : D \to \Sigma  \subset  \mathbb{R}^3$. Then
    \begin{align*}
           \iint_{\Sigma} \omega \, d \varphi   = \iint_D  \Big{[}  P (\varphi (u,v)) &\, dy \wedge dz  \left( \frac{\partial \varphi }{\partial u} , \frac{\partial \varphi }{\partial v}    \right) + Q (\varphi (u,v)) \, dz \wedge dx \left( \frac{\partial \varphi }{\partial u} , \frac{\partial \varphi }{\partial v}    \right)  \\
           &  + R (\varphi (u,v) )\, dx \wedge dy \left( \frac{\partial \varphi }{\partial u} , \frac{\partial \varphi }{\partial v}    \right) \Big{]}dudv \\
            = \iint_D  \Big{[} P (\varphi (u,v))& \left(\frac{\partial y}{\partial u} \frac{\partial z}{\partial v} - \frac{\partial z}{\partial u} \frac{\partial y}{\partial v}   \right)  + Q (\varphi (u,v)) \left(\frac{\partial z}{\partial u} \frac{\partial x}{\partial v} - \frac{\partial x}{\partial u} \frac{\partial z}{\partial v}   \right)  \\
            & + R (\varphi (u,v)) \left(\frac{\partial x}{\partial u} \frac{\partial y}{\partial v} - \frac{\partial y}{\partial u} \frac{\partial x}{\partial v}   \right)  \Big{]} \, dudv \\
            = \iint_D \Big{\langle }  P,Q,R \Big{\rangle}  \,\, \cdot &\,\, \left(  \frac{\partial \varphi}{ \partial u } \times \frac{\partial \varphi}{ \partial v }     \right) \, dudv \\
            = \iint_{\Sigma } \,\,  \langle P, Q, R \rangle \,\, \cdot \, \, & dS
            \end{align*}
        This is the flux of $\langle P, Q, R \rangle $ across $\Sigma$.
\end{example}


\begin{example}[$k=3$, $n=3$] A differentiable $3$-form in $\mathbb{R}^3$ has the formula
\[          \omega  = f(x ,y, z) \, dx \wedge dy \wedge dz    \]
    Given a region  $E \subset \mathbb{R}^3$, the inclusion $ \varphi : E \to \mathbb{R}^3 $ is a parametrized $3$-manifold.  Then
    \[
           \iiint_{E} \omega \, d \varphi    = \iiint_E f(x,y,z)\, dV 
    \]
    This is the usual notion of integral.
\end{example}

\section{Exterior derivatives}

And there is a cool notion of derivatives of forms.





\end{document}

